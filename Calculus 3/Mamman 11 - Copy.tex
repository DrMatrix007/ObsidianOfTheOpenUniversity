# שאלה 1 
![[Pasted image 20241113130523.png]]

 :נתון:
 קיימת פונ $\varphi$ כאשר:
 $$\varphi(B(0^{[k]};1)) = B(a;r)$$
 צ"ל:
 $$\varphi(S(0^{[k]};1)) = S(a;r)$$
 נסתכל על $\varphi$ : (פונקציה אפינית)
$$\varphi(x) = f(x) + b$$
ניקח פונ' 
$$\varphi_{0}(x) = \frac{2}{r}*(f(x) + b - a)$$
כאשר קיימת $A\in M^{\mathbb{R}}_{k\times k}$ כאשר:
$$\varphi_{0}(x) = xA^t+\frac{2}{r}(b-a)$$
לפי הגדרת $\varphi, \varphi_{0}$ נקבל:
$$\varphi_{0}(B(0^{[k]};1)) = B(0^{[k]};2)$$
נסתכל על הבסיס הסטנדרטי $E=(e_{1},e_{2},\dots,e_{k})$ מעל $\mathbb{R}^{k}$, ונקבל:
$$\forall 1\leq n\leq k(\exists x \in B(0^{[k]},1)(\varphi_{0}(x)=e_{n}))$$
לכן $A$ מטריצה הפיכה ולכן הפונקציה
$$f_{0}(x) = xA^t$$
הפיכה ועל, (מהקורס אלגברה ליניארית 1) 
ולכן, ניתן להבין בבירור כי הפונקציה $\varphi_{0}$ הפיכה ועל.
לכן, הפונקציה $\varphi$ הפיכה ועל.
## כיוון 1:
יהי:
$$ c \in S(a,r)$$
לכן קיימות סדרות $(d_{n}),(e_{n})$  כאשר: (לפי הגדרת $\partial B(a;r)$)
$$ \lim_{ n \to \infty } {(d_{n})} = c = \lim_{ n \to \infty } {(e_{n})} $$
$$ \forall n>0 (d_{n} \in B(a,r),e_{n} \notin B(a,r)) $$
נגדיר את הסדרות $(a_{n})_{n},(b_{n})_{n}$ באופן הבא:
$$\forall n>0(a_{n}=\varphi^{-1}(d_{n}),b_{n}=\varphi^{-1}(e_{n}),)$$
נסתכל על הסדרות הללו ונקבל: 
$$\forall n>0(d_{n} \in B(a,r)) \implies \forall n>0(a_{n} \in B(0^{[k]},1))$$
בנוסף, מפני ש $\varphi$ פונ חח"ע,אז לא יכול להיות $b_n$ ,שנמצא מחוץ ל $B(0;1)$, ומקיים $b_n \in B(a;r)$
$$\forall n>0(e_{n} \notin B(a,r)) \implies \forall n>0(b_{n} \notin B(0^{[k]},1))$$
אך ידוע כי מתקיים
$$\lim_{ n \to \infty }(d_{n})=\lim_{ n \to \infty } (e_{n})=c $$
כלומר, (מפני ש $\varphi$ רציפה אז גם $\varphi^{-1}$ רציפה)
$$\lim_{ n \to \infty }(\varphi^{-1}(d_{n})) = \lim_{ n \to \infty }(\varphi^{-1}(e_{n})) = \varphi^{-1}(c)$$
ונקבל
$$\lim_{ n \to \infty } (a_{n}) = \lim_{ n \to \infty }(b_{n}) = \varphi^{-1}(c) $$
לכן, מתקיים:

$$\forall \text{Neighbourhood D of }\varphi^{-1}(c)(\exists n,m(a_{n} \in D\ \text{and}\ b_{m} \in D)) $$
ידוע כי
$$\forall n>0(a_{n} \in B(0^{[k]},1))$$
$$\forall n>0(b_{n} \notin B(0^{[k]},1))$$
ונקבל כי 
$$\varphi^{-1}(c) \in \partial B(0;1)$$
כלומר, 
$$\varphi^{-1}(c) \in S(0^{[k]};1)$$
לכן
$$c \in \varphi(S(0^{[k]};1))$$
ונקבל 
$$S(a;r) \subseteq \varphi(S(0^{[k]};1)) $$
## כיוון 2:
יהי:
$$y \in \varphi(S(0^{[k]};1)),x \in \mathbb{R}^{k}$$
כאשר:
$$\varphi(x) = y$$
ניקח סדרות $(d_{n}),(e_{n})$  כאשר: (לפי הגדרת $\partial B(0^{[k]};1)$)
$$ \lim_{ n \to \infty } {(d_{n})} = x = \lim_{ n \to \infty } {(e_{n})} $$
$$ \forall n>0 (d_{n} \in B(0^{[k]},1),e_{n} \notin B(0^{[k]},1)) $$
מפני ש $\varphi$ פונ' חח"ע, נקבל:
$$\forall n>0(\varphi(d_n) \in B(a;r),\varphi(e_n) \notin B(a;r))$$
מפני ש $\varphi$ פונ' רציפה, נקבל:
$$\lim_{n \rightarrow \infty}(\varphi(d_n)) = \varphi(\lim_{n \rightarrow \infty}(d_n)) = \varphi(x) = y =\varphi(\lim_{n \rightarrow \infty}(e_n))= \lim_{n \rightarrow \infty}(\varphi(e_n))$$
לפי הסדרות $(\varphi(d_n)),(\varphi(e_n))$, ולפי התנהגותן סביב $y$, נקבל כי 
$$\forall \text{Neighbourhood D of }y(\exists n,m(\varphi(d_n) \in D\ \text{and}\ \varphi(e_{m}) \in D)) $$
כאשר
$$ \forall n>0 (d_{n} \in B(0^{[k]},1),e_{n} \notin B(0^{[k]},1)) $$
לכן, לפי הגדרת $y$ נקבל:
$$y \in \partial B(a;r)$$
כלומר,
$$y \in S(a;r)$$
ולכן,
$$\varphi(S(0^{[k]};1)) \subseteq S(a;r)  $$
ולפי הכלה דו כיוונית, נקבל:
$$\varphi(S(0^{[k]};1)) = S(a;r)  $$
מש"ל
# שאלה 2
![[Pasted image 20241116224136.png]]
יהי
$$x \in \partial(A\backslash \partial A)$$
לכן קיימת נקודה $x \in \mathbb{R}^k$ כאשר:
$$\forall \text{Neighbourhood D of }x(\exists a,b \in D(a \in A \backslash \partial A \text{ and } b \notin A \backslash \partial A))$$
ונקבל:
$$\exists \text{Neighbourhood D of }x(\exists a,b \in D((a \in A \text{ and } (b \notin A \text{ or } b \in \partial A))))$$
יהי סביבה $D$ של $x$. קיים $b \in D$ כאשר $b \notin A$ או $b \in \partial A$
## אפשרות אחת: $b \notin A$
קיימת נקודה מחוץ ל$A$, ולכן יש איבר ב$A$ ואיבר מחוץ ל$A$
## אפשרות שנייה: $b \in \partial A$
לפי הגדרת $b$, לכל סביבה של $b$, כולל $D$, ולכן יש איבר ב$A$ ואיבר מחוץ ל$A$
לכן, לכל סביבה של x מתקיים:
$$\forall \text{Neighbourhood D of }x(\exists a,b \in D(a \in A \text{ and } b \notin A))$$
לכן מתקיים
$$ x \in \partial A $$
לכן
$$\partial ( A \backslash \partial A ) \subseteq \partial A $$
מש"ל


# שאלה 3
![[Pasted image 20241116232851.png]]
## כיוון 1
נניח כי $U$ קבוצה פתוחה.
נניח בשלילה כי קיים $a \in \mathbb{R}$ כאשר מתקיימת ההגדרה הנגדית לגבול:
$$\exists \varepsilon>0(\forall \delta>0(\exists a<x<a+\delta(f(x)-f(a) \ge \varepsilon)))$$
אראה כי קיימת נקודה $c \in U$ כאשר:
$$\exists c\in U(\forall \text{Neighbourhood D of }c \exists a,b \in \mathbb{R}^k(a \in U \text{ and } b \notin U))$$
מהנחת השלילה, נקבל:
$$\exists \varepsilon>0(\forall \delta>0(\exists a<x<a+\delta(f(x) \ge f(a)+\varepsilon)))$$
לפי הערכים מהנחת השלילה, ניקח את הנקודות: 
$$(a,f(a)+\varepsilon) \in U$$
$$(a,f(a)+\varepsilon-\frac{\delta}{2}) \in U$$
$$f(x) \ge f(a)+\varepsilon \implies (x,f(a)+\varepsilon) \notin U$$
נקבל שלכל $\delta>0$, המרחק בינהן הוא:
$$d = \sqrt{(a-x)^2+(f(a)+\varepsilon - f(a) - \varepsilon)^2}$$
$$d = |x-a|$$
$$0 < x-a < \delta$$
לכן, לכל סביבה של הנקודה $(a,f(a)+\varepsilon)$
קיים $\delta > 0$ כאשר הנקודות $(x,f(a)+\varepsilon),(a,f(a)+\varepsilon-\frac{\delta}{2})$ נמצאת בסביבה זו.
לכן $c$ מקיימת
$$\forall \text{Neighbourhood D of }c \exists a,b \in \mathbb{R}^k(a \in U \text{ and } b \notin U))$$
לכן $U$ לא פתוחה. לכן קיבלנו סתירה.
ולכן, הגבול מתקיים לכל $a$.
## כיוון 2
נניח כי מתקיים:
$$\forall a(\ \lim_{x->a^+}{f(x)}=f(a))$$
יהי $a \in \mathbb{R}$
ידוע כי מתקיים:
$$\forall \varepsilon>0(\exists \delta>0(a<x<a+\delta \implies |f(x)-f(a)| < \varepsilon))$$
מפני ש $x>a$, ו$f$ פונ' מונוטונית עולה, נקבל כי מתקיים:

$$\forall \varepsilon>0(\exists \delta>0(\forall a-\delta<x<a+\delta (f(x) < f(a)+\varepsilon)))$$
יהי $\varepsilon>0$. לכן קיים $\delta>0$ כאשר $\forall a<x<a+\delta (f(x) < f(a)+\varepsilon)$
ניקח את הנקודה $(a,f(a)+2\varepsilon)$. 
אראה כי קיימת סביבה לנקודה זו המוכלת ב $U$:
ניקח את הקבוצה הפתוחה:
$$U_a = \{(x,y)|\ y>f(x) \text{ and } a-\delta<x<a+\delta \}$$
ידוע כי לכל $a-\delta<x<a+\delta$ מתקיים $f(x)<f(a)+\varepsilon$
לכן,
$$(a-\delta,a+\delta) \times (f(a)+\varepsilon,\infty) \subseteq U_a$$
לפי טענה 2.א.29 מכרך א' בספר הלימוד,
נקבל כי הקבוצה $(a-\delta,a+\delta) \times (f(a)+\varepsilon,\infty)$
